\documentclass[paper=A4,BCOR=8.00mm,DIV=1]{scrbook}				%DIV=calc berechnet ausgehend von Schriftgrosse etc selbst einen sinnvollen wert
%  headsepline=true,headinclude=false,				% Horizontale Linie unter Kopfzeilen
%  footsepline=false,footinclude=false,				% Horizontale Linie ueber Fusszeilen
%  pagesize,							% Diese Option uebergibt dem pdf die Seitengroesse
%  fontsize=12,chapterprefix=true,twoside,draft=false,		%
%  headings=normal,						% Hiermit wird fuer kleine Ueberschriften gesorgt
%toc=bib,toc=graduated,					% Eintrag fuer bibliography in table of contents
%								% Eintraege einruecken im toc
%  numbers=noenddot,						% Endpunkt bei chapter1. und 1.1. entfernt




%  parskip=half+
%  %cleardoublepage=plain% sorgt dafuer, dass leere Seiten nummeriert werden

																		% fuer definecolor
%\hyphenation{nan-o-struc-ture}
%*******************************************************************************
% eigene Pakete und Einstellungen etc.
%===============
\usepackage[ngerman,english]{babel} 			% zwei Sprachen benutzen!
\usepackage[T1]{fontenc}									% T1 Kodierung anschalten, um alle 256 Europa-typischen Buchstaben zu bieten 
																					% z.B. Umlaute
\usepackage[utf8]{inputenc}
\usepackage[scaled]{helvet}
%\usepackage[bitstream-charter]{mathdesign}		% Hier wird die Schriftart fuer den Fliesstext gewaehlt
\usepackage[sort,round,authoryear]{natbib} 		% sortiert, runde Klammern, Autor-Jahr-Literaturverweise, 
%\usepackage{setspace}
%\usepackage{mathptmx}        				%Postscript fonts
\usepackage{array}      				% fuer die einzige Tabelle
\usepackage[final]{graphicx}   				% fuer die Bilder, [draft] -> nur Rahmen statt Bilder
\usepackage{epsfig}					% fuer Grafiken auf der Titelseite
\usepackage{xcolor}					% fuer \definecolor{rgb}{R,G,B}
\usepackage{amsmath, amsthm, amssymb, amsfonts}		% Mathesatz
%\usepackage[format=plain,labelfont=bf,normal,font=sf]{caption} % up, it, sl, sc,md, bf, rm, sf, or tt sets the font attribute of the caption labels.
\usepackage[position=top,singlelinecheck=false]{subfig}	
\usepackage{hyperref}
%\usepackage{cite}					% improve numerical citations, dont use with natbib (natbib uses sort instead)
\usepackage[german,disable]{todonotes}				% insert todo-notes, [disable] -> todo notes werden abgeschaltet
\usepackage{siunitx}					% enables input of units by name in formulas (e.g. \meter gives m)
\usepackage{lipsum}					% for lipsum-text to test layout
\usepackage{url}					% besserer Zeilenumbruch bei urls in der bibliography
\usepackage[utf8]{inputenc}		% utf8 unter Linux besser als T1 oder latin9 (auf mac-os: macce)
\usepackage{tikz}					% package for drawings (not used)
\usepackage{listings}					% package for source code listings
\usepackage{textcomp}
\usepackage{units}
\usepackage{sidecap} \sidecaptionvpos{figure}{t}  %\usepackage[option]{sidecap}:outercaption (default)
\usepackage{bbold}
%\usepackage{listings} \lstset{numbers=left, numberstyle=\tiny, numbersep=5pt} \lstset{language=Python} 

%\usepackage{pdflscape}					% package for landscape pages

%\usepackage{extramarks}

\renewcommand{\encodingdefault}{T1}
\renewcommand{\rmdefault}{cmr}		% Computer Modern as serif font
%\renewcommand{\sfdefault}{lhv}		% TUM Helvetica as sans serif font
%\renewcommand\@ptsize{12}
%\renewcommand{\rmdefault}{cmr}

\setlength{\parskip}{0pt}
\setlength{\parindent}{0pt}
%\parindent 0em 						% Einzug bei neuem Absatz
\setcapindent{0.em} 					% Hier kann man den Einzug bei mehrzeiligen captions waehlen
\graphicspath{{pics/}} 					%sucht Grafiken im Verzeichnis pics und dessen Unterverzeichnisse (//)
% eigene Farben
%===============
\definecolor{tumblauTitel}{rgb}{0,.459,.737}			% TUM-Blau 0 101 189
\definecolor{tumgruen}{rgb}{.4,.6,.11}				% TUM-Gruen 103 154 29

% Stile aendern
%===============
\addtokomafont{chapter}{\normalfont\sffamily\color{tumblauTitel}\fontsize{25}{33}\selectfont} 		% Vorschalten von \normalfont schaltet ab, dass die Ueberschriften fett gesetzt werden
\addtokomafont{section}{\normalfont\sffamily\color{tumblauTitel}\fontsize{17.28}{20}\selectfont}
\addtokomafont{captionlabel}{\normalfont\sffamily\fontsize{12}{12}\selectfont}
\addtokomafont{caption}{\itshape}		% caption of figures, tables
%\setstretch{1.2}											% Um beliebige Werte des Duchschusses zu wÀhlen

%  Abkuerzungen und eigene Befehle
%===============
\renewcommand{\d}{\text{d}}		% Differentialoperator in Formeln nicht italic
\newcommand{\bra}[1]{\left\langle #1\right|}
\newcommand{\ket}[1]{\left|#1\right\rangle}
\newcommand{\braket}[2]{\left\langle #1|#2\right\rangle}
\renewcommand{\vec}[1]{\boldsymbol{#1}}										% typeset vectors in bold face
\newcommand\relphantom[1]{\mathrel{\phantom{#1}}}
\newcommand{\opbf}[1]{\hat{\mathbf{#1}}}							% operator bold face
\newcommand{\Pbn}{\mathrm{P}_\mathrm{b0}}			% P b null
\newcommand{\edP}{^\mathrm{31}\mathrm{P}}			% einundreissig P
\newcommand{\natSi}{^\mathrm{nat}\mathrm{P}}			% natural P
\newcommand{\Tm}{$T_\mathrm{M}$}
%\newcommand{\NV-}{$NV^{-}$ center}
\newcommand{\cuu}{c_{\uparrow\uparrow}}
\newcommand{\cud}{c_{\uparrow\downarrow}}
\newcommand{\cdu}{c_{\downarrow\uparrow}}
\newcommand{\cdd}{c_{\downarrow\downarrow}}

\newcommand{\mean}[1]{\langle #1 \rangle}

\newcommand{\muB}{\mu_\mathrm{B}}
%\DeclareUnicodeCharacter{2107}{\Efield}

\newcommand{\Rphitwo}{
\left(\begin{matrix}
  (R_{\Phi,2})_{11} & (R_{\Phi,2})_{12} \\
  (R_{\Phi,2})_{21} & (R_{\Phi,2})_{22}
\end{matrix}\right)
}
\newcommand{\circled}[1]{\raisebox{.5pt}{\textcircled{\raisebox{-.9pt}{#1}}}}

% Konfiguration listings
\lstset{
    showstringspaces=false,
    %stringstyle=\color{yellow}	% boeser Befehl, macht bei single quotes fehlermeldung...
    %basicstyle=\scriptsize\ttfamily,
    basicstyle=\tiny\ttfamily,
    language=Matlab,
    numbers=left,
    numberstyle=\tiny,
    firstnumber=0,
    stepnumber=10,
    keywordstyle=\color{tumblauTitel},
    commentstyle=\color{tumgruen},
    breaklines=true,
    upquote=true
   % emph={for}, emphstyle=\color{red}
}

\newcommand{\executeiffilenewer}[3]{%
\ifnum\pdfstrcmp{\pdffilemoddate{#1}}%
{\pdffilemoddate{#2}}>0%
{\immediate\write18{#3}}\fi%
}
\newcommand{\includesvg}[2]{%
\executeiffilenewer{#1.svg}{#1.pdf}%
{inkscape -z -D --file=#1.svg %
--export-pdf=#1.pdf}%
\includegraphics[#2]{#1.pdf}
}

\recalctypearea			% Die Berechnung des Satzspiegels muss nach!! der Wahl der Schrift und des Durchschusses geschehen
%*******************************************************************************

\begin{document}

% Titelseiten
\def\logooffset{271}
\def\fakultaetlogo{\epsfig{file=../pics/logos/PH.pdf,height=9.605mm}}
\def\wsilogo{\epsfig{file=../pics/logos/wsiLogo.pdf,height=9.605mm}}
\def\tumlogo{\epsfig{file=../pics/logos/tumlogo.pdf,height=9.605mm}}
\def\deadline{August 2017}
\def\title{Electrical Readout of NV$^{-}$ Centers in\\ Diamond}	% $_\mathsf{2}$
\def\author{Felix Hartz}
\def\typeOfThesis{Master Thesis}

\newlength{\addresstop}
\newlength{\titletop}
\newlength{\firstpagespace}
\newlength{\titleskip}

\renewcommand\maketitle{
\begin{titlepage}
  {\noindent\setlength{\unitlength}{1mm}%
  \begin{picture}(0,0)(25,\logooffset)
    \put(25,272){\fakultaetlogo}	% 25,272
    \put(40,272){\wsilogo}
    \put(176,272){\tumlogo}				%176,272
  \end{picture}}%
  %
  \setlength{\addresstop}{4.44cm}
  %
  \setlength{\titletop}{9.5cm}
  %
  \setlength{\firstpagespace}{4.44cm-1in-\headsep-\headheight+\baselineskip}
  \vspace*{\firstpagespace}%

  \fontsize{12}{16}\selectfont\sffamily Walter Schottky Institut\\
  Center for Nanotechnology and Nanomaterials \\ %Zentralinstitut f\"ur physikalische Grundlagen der Halbleiterelektronik \\
  Fakult\"at f\"ur Physik der Technischen Universit\"at M\"unchen

% 
   
   \setlength{\titleskip}{\titletop-\addresstop-4\baselineskip}
   \vspace{\titleskip} 
%   % Anpassung der Schriftgroesse des Titels exakt 15 - Durchschuss 20
  \fontsize{20.74}{26}\selectfont\sffamily{\textcolor{tumblauTitel}{\title}}\par\smallskip
  \vspace{2cm}
  \fontsize{12}{16}\selectfont\textbf{\author}\par\smallskip
  \vspace{1cm}
  \fontsize{12}{16}\selectfont\sffamily\typeOfThesis\par\smallskip
%   \vspace{20\baselineskip}
    \vfill
   \deadline
\end{titlepage}
}

\newcommand{\emptypage}{
  \mbox{}
  \thispagestyle{empty}
  \newpage
}

\newcommand{\Ehrenwort}{
\markright{Erkl\"arung der Selbstst\"andigkeit}
\cleardoublepage
\section*{Erkl\"arung der Selbstst\"andigkeit}

\vspace{\baselineskip}
Ich erkl\"are hiermit, dass ich die vorliegende Arbeit selbstst\"andig und ohne Benutzung anderer als der angegebenen Hilfsmittel und Quellen angefertigt habe.

\vspace{2cm}

\hspace{1.5cm} Garching, den  \hspace{5cm} \author
}
\frontmatter		% arabische Nummerierung
\maketitle
% %\mbox{}
\emptypage

% Abstract
%================
\selectlanguage{ngerman}
%\include{chapters/zusammenfassung}

\selectlanguage{english}

% % \clearpage

% table of contents
%================
\tableofcontents

% chapters
%================
\mainmatter			% roemische Nummerierung

\include{chapters/intro/intro}
\include{chapters/basics/basics}
\include{chapters/samples/samples}
\include{chapters/setup/setup}
\include{chapters/experiments/experiments}
\include{chapters/summary/summary}

% appendix
================
\begin{appendix}
\include{chapters/appendix/appendix}
\label{appendix}
\end{appendix}

% Literaturverzeichnnis 
%================
%Literaturverzeichnis mit BibTeX generieren
\backmatter

\bibliographystyle{maxbib110925}	%unsrtnat, chicago, abbrvnat, plainnat, maxbib2, maxbib110925
\bibliography{literatur/literatur}

% Danksagung, Erklaerung der Selbststaendigkeit 
%================
% %\cleardoublepage
% 
\selectlanguage{ngerman}
%\include{chapters/danksagung/danksagung}
%\Ehrenwort
\end{document}
% ------------------------------------------------------------------------ 
